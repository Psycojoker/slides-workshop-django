\documentclass{beamer}
%\usepackage{unicode}
%\usetheme{default}
\begin{document}
\title{Workshop django}

\maketitle{}

\begin{frame}{Requirements}
\begin{itemize}
    \item savoir coder
    \item programmation orienté objet
    \item python (la base)
    \item sql (pas forcement poussé)
    \item html (la base)
    \item regexp (la base)
    \item savoir utiliser un shell (la base)
\end{itemize}
\end{frame}

\begin{frame}{Introduction}
    C'est quoi django ?
    \begin{itemize}
        \item framework web en python\pause
        \item full stacked (vs microframework comme flask, bottle et les 15 milles autres)\pause
        \item vous "impose" un orm et un système de templates (changeable tous les 2 mais pas facilement)\pause
        \item mais en échange vous avez les django app
    \end{itemize}
\end{frame}

\begin{frame}{Installation}
2 façons de faire
\begin{itemize}
    \item la classique avec le pkg manager de la distribution (eg: sudo apt-get install django)
    \item en utilisant pypi via pip (on va utiliser celle là) dans un virtualenv
\end{itemize}
\end{frame}

\begin{frame}{Installation}
    Faire:
    \begin{itemize}
        \item Installation de pip (sudo apt-get install python-pip)\pause
        \item sudo pip install virtualenv (ou alors avec le package python-virtualenv)\pause
        \item aller dans un nouveau répertoire (mkdir blog \&\& cd blog)\pause
        \item virtualenv --no-site-packages --distribute ve\pause
        \item source ve/bin/activate \# pour rentrer dans le venv\pause
        \item pip install django\pause
        \item deactivate \# on doit faire ça pour que le bin django-admin soit prit en compte
        \item source ve\/bin\/activate
    \end{itemize}
    \vspace{3mm}
    \pause
    Si vous voulez sortir du venv:
    \begin{itemize}
        \item deactivate
    \end{itemize}
\end{frame}

\begin{frame}{Mon premier projet django}
    Faire dans le virtualenv:
    \begin{itemize}
        \item django-admin.py startproject <nom du projet>
    \end{itemize}
    \pause
    Lancer le serveur de développement:
    \begin{itemize}
        \item cd <nom du projet> \&\& python manage.py runserver
    \end{itemize}
    Puis aller sur http://0.0.0.0:8000
\end{frame}

\begin{frame}
    Vous allez obtenir une hiérarchie du style:
%./project/urls.py
%./project/settings.py
%./project/__init__.py
%./project/wsgi.py
%./manage.py
\end{frame}

\begin{frame}
    Remarque: vous allez aussi rencontrer des projets ayant la forme suivante (django <= 1.3)
%./urls.py
%./settings.py
%./manage.py
%./__init__.py
\end{frame}

%\begin{theorem}
  %In a right triangle, the square of hypotenuse equals
  %the sum of squares of two other sides.
%\end{theorem}

%*Urlresolver
%*Hello World (HttpResponse)
%*définir des urls simple
%*choper des arguments
%*choper des arguments només

%*views.py
%*utiliser un string dans urlresolver

%*Templates
%*settings.py
%*template bidon
%*comment appeler un template version moyen âge
%*render (render_to_content)

%*présentation du language de templates
%*{{ }}
%*filtres
%*
%*

%*Modèles
%*premier modèle
%*syncdb + settings.py
%*warning sur les modifs de db (évoquer south)
%*liste des champs classique, pointer sur la doc de django
%*shell
%*créer un modèle, les 2 façons (.create, parler du .save)
%*accèder à un/des modèle (.all, .filter, .get + chainage)
%*modifier un modèle
%*supprimer un/des modèles
%*.update

%*Interface admin
%*ça existe
%*comment rajouter un modèle
%*vais pas trop en parler, y a plein de trucs cool à faire mais pas le sujet

%*Apps
%*principe, pourquoi c'est cool, organisation, réutilisabilité
%*où en trouver ? github + google + http://www.djangopackages.com
%*mater ça http://youtu.be/A-S0tqpPga4
%*python manage.py startapp
%*bouger tout ce qu'on a fait dedans
%*attention à la db ! (les modèles changent) + !! settings.py installed apps
%*comme gérer dans urls.py
%*import vs strings vs include

%*Avancé
%*get_object_or_404
%*
%*forms
%*héritage templates

%*Parler de ce dont j'ai pas parlé
%*les middlesware
%*templateloarders
%*templatestags
%*commandes perso
%*generic views ?
%*request.user
%*login_required
%*i18n
%*fixtures

%*jquery
%*introduction
%*firebug
%*$.document.ready vs $() vs autre brole
%*selectors
%*.hide
%*.show
%*events (onclick)
%*.html
%*.append
%*.data
%*AJAX $.get $.post

%Repomper ça http://sametmax.com/schema-de-fonctionnement-general-de-django/

\end{document}
